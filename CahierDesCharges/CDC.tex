\documentclass[12pt]{article}
\usepackage[top=2cm, bottom=2cm, left=2cm, right=2cm]{geometry}
\usepackage{fancyhdr}
\usepackage{graphicx}
\usepackage{hyperref}
\usepackage[utf8]{inputenc}
\usepackage[T1]{fontenc}
\setlength{\parindent}{0pt}
%\usepackage{hyperref}  %table des matières contient des
						%liens pour aller directement au section

\begin{document}
\begin{titlepage}
\begin{center}

\includegraphics[scale=0.4]{logo.png}\\[3cm]

{\huge Cahier des charges}\\[1cm]

\rule{\linewidth}{0.5mm} \\[0.4cm]
{ \huge \bfseries Amélioration d’un programme de cartographie par séquençage: création d’une interface
d’entrée/sortie. \\[0.4cm] }
\rule{\linewidth}{0.5mm} \\[1.5cm]

\noindent
\begin{minipage}{0.4\textwidth}
  \begin{flushleft} \large
    \emph{Auteurs :}\\
    Hermes \textsc{Paraqindes}\\
    Juliette \textsc{Geoffray}\\
    Eric \textsc{Cumunel}
  \end{flushleft}
\end{minipage}%
\begin{minipage}{0.4\textwidth}
  \begin{flushright} \large
    \emph{Encadrants :} \\
    Fabrice \textsc{Besnard}\\
    Laurent \textsc{Gueguen}\\
  \end{flushright}
\end{minipage}\\[4cm]

{\large 12 Février 2018}

\end{center}
\end{titlepage}

\clearpage
\tableofcontents

\newpage

\section{Présentation du projet}
\subsection{Contexte}
\subsection{Objectifs}
\subsection{Description de l'existant}
\subsection{Critères d'acceptabilité du produit}
\section{Expression des besoins}
\subsection{Besoins fonctionnels}
\subsection{Besoins non fonctionnels}
\section{Contraintes}
\subsection{Délais}
\subsection{Autres contraintes}
\section{Déroulement du projet}
\subsection{Planification}
\subsection{Plan d'assurance qualité}
\subsection{Documentation}
\subsection{Responsabilités}
\subsubsection{Maîtrise d'ouvrage}
\subsubsection{Maîtrise d’œuvre}
\section{Bibliographie}

\end{document}
